%%%%%%%%%%%%%%%%%%%%%%%%%%%%%%%%%%%%%%%%%%%%%%%%%%%%%%%%%%%%%%%%%%%%
%% I, the copyright holder of this work, release this work into the
%% public domain. This applies worldwide. In some countries this may
%% not be legally possible; if so: I grant anyone the right to use
%% this work for any purpose, without any conditions, unless such
%% conditions are required by law.
%%%%%%%%%%%%%%%%%%%%%%%%%%%%%%%%%%%%%%%%%%%%%%%%%%%%%%%%%%%%%%%%%%%%

\documentclass[
  digital,     %% The `digital` option enables the default options for the
               %% digital version of a document. Replace with `printed`
               %% to enable the default options for the printed version
               %% of a document.
%%  color,       %% Uncomment these lines (by removing the %% at the
%%               %% beginning) to use color in the printed version of your
%%               %% document
  twoside,     %% The `twoside` option enables double-sided typesetting.
               %% Use at least 120 g/m² paper to prevent show-through.
               %% Replace with `oneside` to use one-sided typesetting;
               %% use only if you don’t have access to a double-sided
               %% printer, or if one-sided typesetting is a formal
               %% requirement at your faculty.
  lof,         %% The `lof` option prints the List of Figures. Replace
               %% with `nolof` to hide the List of Figures.
  lot,         %% The `lot` option prints the List of Tables. Replace
               %% with `nolot` to hide the List of Tables.
]{fithesis4}
%% The following section sets up the locales used in the thesis.
\usepackage[resetfonts]{cmap} %% We need to load the T2A font encoding
\usepackage[T1,T2A]{fontenc}  %% to use the Cyrillic fonts with Russian texts.
\usepackage[
  main=czech, %% By using `czech` or `slovak` as the main locale
                %% instead of `english`, you can typeset the thesis
                %% in either Czech or Slovak, respectively.
  english, german, russian, czech, slovak %% The additional keys allow
]{babel}        %% foreign texts to be typeset as follows:
%%
%%   \begin{otherlanguage}{german}  ... \end{otherlanguage}
%%   \begin{otherlanguage}{russian} ... \end{otherlanguage}
%%   \begin{otherlanguage}{czech}   ... \end{otherlanguage}
%%   \begin{otherlanguage}{slovak}  ... \end{otherlanguage}
%%
%% For non-Latin scripts, it may be necessary to load additional
%% fonts:
\usepackage{paratype}
\def\textrussian#1{{\usefont{T2A}{PTSerif-TLF}{m}{rm}#1}}
%%
%% The following section sets up the metadata of the thesis.
\thesissetup{
    date        = \the\year/\the\month/\the\day,
    university  = mu,
    faculty     = phil,
    type        = mgr,
    programme   = Informační studia a knihovnictví,
    field       = Informační studia a knihovnictví,
    department  = Katedra informačních studií a knihovnictví,
    author      = Ing. Pavel Dytrych,
    gender      = m,
    advisor     = {Ing. Filip Janovič PhD., MBA},
    title       = Zavádění datového skladu v rámci Masarykovy univerzity,
    TeXtitle    = Zavádění datového skladu v rámci Masarykovy univerzity,
    keywords    = {keyword1, keyword2, ...},
    TeXkeywords = {keyword1, keyword2, \ldots},
    thanks      = {%
      Na tomto místě bych rád poděkoval:
    },
    bib        = bibliografie.bib,
    %% The following keys are only useful, when you're using a
    %% locale other than English. You can safely omit them in an
    %% English thesis.
    programmeEn        = NA,
    fieldEn            = Cognitive Sciences,
    departmentEn       = Department of Psychology,
    titleEn            = What Can Typography Tell Us
                 About the Nature of Man,
    TeXtitleEn         = What Can Typography Tell Us
                 About the Nature of Man,
    keywordsEn         = {keyword1, keyword2, ...},
    TeXkeywordsEn      = {keyword1, keyword2, \ldots},
    abstractEn         = {%
    This is the English abstract of my thesis, which can

      span multiple paragraphs.
    },
}
\usepackage{makeidx}      %% The `makeidx` package contains
\makeindex                %% helper commands for index typesetting.
%% These additional packages are used within the document:
\usepackage{paralist} %% Compact list environments
\usepackage{amsthm}
\usepackage{amsfonts}
\usepackage{url}      %% Hyperlinks
\usepackage{markdown} %% Lightweight markup
\usepackage{listings} %% Source code highlighting
\lstset{
  basicstyle      = \ttfamily,
  identifierstyle = \color{black},
  keywordstyle    = \color{blue},
  keywordstyle    = {[2]\color{cyan}},
  keywordstyle    = {[3]\color{olive}},
  stringstyle     = \color{teal},
  commentstyle    = \itshape\color{magenta},
  breaklines      = true,
}
\usepackage{floatrow} %% Putting captions above tables
\floatsetup[table]{capposition=top}
\usepackage[babel]{csquotes} %% Context-sensitive quotation marks
\begin{document}
%% The \chapter* command can be used to produce unnumbered chapters:
\chapter{Úvod}
Velký rozmach informačních technologií v druhé polovině dvacátého století vedl k rozsáhle automatizaci a zjednodušení mnoha procesů, u kterých čím dál více úkonů přecházelo z lidských zdrojů na informační systémy. Procesy, které jsou realizovány za pomoci informačních systémů umožňují lepší optimalizaci, nepřetržitý provoz a také z velké části eliminují chyby, které jsou způsobeny lidským faktorem. Z těchto důvodů se postupně IT služby prosadili ve všech aspektech lidské společnosti, která je nyní na jejich bezchybném fungování a dostatečné dostupnosti de facto závislá. 

Informační služby za nás dnes vykonávají velké množství činností, z nichž by velká část nebyla nebyla ani realizovatelná, pokud bychom k realizaci využili pouze lidské zdroje. Díky informačním službám je dnes také možné uchovávat a zpracovávat obrovské množství dat, které následně můžeme využívat jako znalostní bázi pro naše rozhodování, ať co se světa velkých organizací a korporací týče, tak i v běžném životě jednotlivců.

Spolu s tímto rozmachem byl nutný i vznik patřičných standardizovaných postupů, za pomoci kterých můžeme tyto služby v rámci organizací zavádět a udržovat, neboť samotný vývoj IT služeb je značně odlišný od jejich implementace a údržby. 

Tato diplomová práce se zabývá rozborem procesu zavádění IT služeb v rámci Ústavu výpočetní techniky (ÚVT) Masarykovy univerzity, a to konkrétně na příkladu zavádění služby datového skladu. Diplomová práce postupně danou problematiku rozebírá jak z teoretického, tak z aplikačního hlediska. Teoretická část zahrnuje potřebné ukotvení v problematice datových skladů a managementu informačních služeb, jejich historický vývoj, kategorizaci, základní popise a charakteristiku. Velká část je pak věnována popisu procesu zavádění datového skladu a konkrétního případu využití. Tento proces následně prochází teoretickou evaluací a případnou optimalizací. 

V aplikační části se pak diplomová práce věnuje samotnému technickému řešení datového skladu a jednoho ukázkového případu jeho využití. Na základě výsledků evaluace zaváděcího procesu z teoretické části, je pak navržen zaváděcí proces pro ukázkový případ, který je pak v rám diplomové práce nasazen. 

Cílem diplomové práce je náhled na proces zavádění nové IT služby v rámci ÚVT MUNI a to jak z technického hlediska, tak z hlediska managementu služeb z pohledu frameworku ITIL. Praktickým výstupem pak je uživatelský report, který je vytvořen anonymizovanými daty z datového skladu a nasazen za pomocí procesu, který splňuje požadavky frameworku ITIL.
\chapter{Datový sklad}

V dnešní společnosti se každý den vygeneruje ohromné množství dat a informací a
toto množství neustále roste. Zatímco ve starém Římě dosahovaly, při přepočtu na dnešní
jednotky, největší sbírky svitků velikosti okolo 100Mb \parencite[p.~157]{Smil2021}, v roce 2016 již
lidstvo dokázalo vygenerovat 16Zb dat každý rok\parencite[p.~160]{Smil2021}, což je nárůst
o celých 14 řádů.

Spolu s velkým množstvím nově vygenerovaných informací se ale mění jejich
forma. Ve zmíněných římských sbírkách, bychom nalezli zejména písemnosti, ty však v současné době představují s
píše marginální čast uložených informací. Například v repozitářích Kongresové knihovny ve Spojených státech amerických, zabírají písemnosti
pouze 1 \% ze všech uchovaných informačních artefaktů. \parencite[p.~158]{Smil2021}
Vzhledem k takto velkému množství dostupných informací, je jejich zpracovaní značně
komplikované, a to i na úrovni menších celků, jako jsou například obchodní společnosti, univerzity a další podobné organizace.Pokud však chceme na základě informací dělat jakékoliv analýzy čí rozhodnutí, je velmi vhodné informace zkonsolidovat do jednoho úložiště,
provést základní očištění a data alespoň základně standardizovat.

Jednou z možností, jak tohoto docílit, je vytvoření datového skladu. Pojem datový
sklad může mít vícero definic. Je možné na něj nahlížet například pouze jako na proces, který má
zjednodušit přístup k datům a kdy je technické řešení datového skladu pouze podpůrný prvek celého procesu.
Nejčastěji se však na datový sklad pohlíží jako na systém, který \emph{extrahuje, čistí, upravuje a dodává zdrojová data do dimenzionálního
datového úložiště a poté podporuje a zavádí dotazování a analýzu dat za účelem
rozhodování.} \parencite[s.~23]{Kimballc2004}

Z této definice jasně vyplývá, že primárním účelem datového skladu je centralizace a
standardizace dat, díky které je pak přístup k datům usnadněn. Zároveň je zde důležité
uvést, že každý datový sklad se skládá z několika komponent, z nichž nejdůležitější je proces
extrakce, čištění a úpravy dat. Tento proces se ve spojitosti s datovými sklady nazývá ETL
proces \footnote{z angl. Extract, Transform and Load} a pojem datový sklad tak nelze zaměňovat
s pouhým datový úložištěm. \parencite[s.~24]{Kimballc2004}

Využití datových skladů je pak velmi rozmanité, jednou z nejčastějších aplikací je ale
jejich využití ve spojení se systémy podpory rozhodování \footnote{zkráceně DSS z angl. Decision support
systém}, kde často slouží jako primární znalostní báze. \parencite[s.~2]{Inmon2005}

\section{Historický vývoj datových skladů}
Datové sklady se spolu se systémy podpory rozhodování začaly vyvíjet v 60. letech,
kdy z důvodu ukládání dat na magnetické pásky bylo problematické data zpětně procházet a
analyzovat. Data se tak postupně začala pročišťovat a ukládat v již zpracované formě, což by
se dalo označit za prvopočátek datových skladů.\parencite[s.~2]{Inmon2005}

V polovině 60. let, se postupně začala rozšiřovat technologie diskových úložišť. Ta,
na rozdíl od magnetický pásků, umožňovala přímý přístup k datům bez nutnosti jejich
kompletního nahrání do pamětí počítače. Díky této technologii začali vznikat první systémy,
které lze označit za DBMS \footnote{z angl. Database management system}, které umožňovaly lepší
správu dat.\parencite{Foote19042018} Vzhledem k primitivnímu systému uložení dat a jejich nízké strukturalizaci 
však tyto systémy neposkytovali data v takové kvalitě jako je známe dnes.

Velkým milníkem byly relační databáze, které začaly vznikat v osmdesátých letech
a které pomocí dotazovacích jazyků, jako je například SQL \footnote{z angl. Structured Query
Language} umožňovaly, v porovnání s předchozím stavem, velmi snadnou práci s daty.\parencite{Foote19042018}

Spolu s nástupem osobních počítačů a rostoucí velikostí firemních sítí, stoupala i
potřeba skutečného datového skladu. Vzájemně propojené osobní počítače měly oproti
architektuře centrálních počítačů velkou nevýhodu v roztříštěnosti dat, kdy data byla uložena
na více míst a bylo složité udržovat data aktuální a konzistentní. Souhrnně se
tento problém nazývá \emph{spider web}. \cite[s.~6]{Inmon2005} Nástup osobních počítačů, a s 
ním spojený rozvoj specializovaných programů jako byl například Microsoft Excel, Microsoft Access atp., 
přinesl další problém a to neustále se zvětšující množství datových formátů, ve kterých byla data uložena.\parencite{Foote19042018}

V devadesátých letech, tak všechny tyto skutečnosti vyvrcholily v potřebu vzniku
jednotného úložiště, které by umožňovalo data snadno udržet v aktualizované a konzistentní
podobě. Za tímto účelem vzniklo několik konceptů, z nich nejrozšířenější je již výše zmíněný koncept
datového skladu a následně koncepty data lake a data cube.

Koncept data lake, se od datového sklad odlišuje převážně tím, že data
nepřevádí do jednotné struktury, ale udržuje je v jejich původní formě. Tzn. jedná se
primárně o jednotné úložiště, kde mohou být data uložena ve všech možných souborových
i datových formátech a kde jsou k datům doplněny detailní anotace, které usnadňují jejich následné zpracování.\parencite{Foote19042018}
Data tedy v tomto případě neprochází žádnou transformací, což komplikuje jejich využití, velkou výhodou však je, že  díky chybějící 
transformaci dosahuje mnohem vyšších rychlostí při manipulaci s daty, díky čemuž je tento koncept vhodnější pro aplikace,
zpracovávající velké množství dat s ohledem na čas.\parencite[s.~1]{Harby20221217}

Koncept data cube je odlišný. Ukládá data ve třech a vícerozměrných maticích, kde každá matice představuje jednu datovou dimenzi. Když
data v dimenzích následně relačně propojíme, je možné snadno filtrovat data dle
jednotlivých dimenzí, bez nutnosti propojovat tabulky mezi sebou.\parencite{Foote19042018}

Všechny tři zmíněné koncepty mají dodnes své opodstatnění a jsou i nadále využívány. S ohledem na již zmíněný masivní nárůst zpracovávaných informací, a s tím spojenými nároky na rychlost jejich zpracování, se však začíná prosazovat nový koncept, a to koncept data lakehouse. Jak již název napovídá, koncept data lakehouse kombinuje výhody úložišť typu data lake a data warehouse. Data lakehouse je schopné, stejně jako data lake, zpracovávat surová data velkou rychlostí a flexibilně, zároveň však poskytuje úložiště i pro vyčištěná a strukturovaná data, stejně jako data warehouse. U data lakehouse se také počítá s přímým napojením analytických nástrojů sloužících například pro BI\footnote{z angl. Business Intelligence}, či jako informační báze pro AI\footnote{z angl. Artificial Intelligence}.

V rámci Masarykovy univerzity byl nasazen koncept datového skladu, tak jak byl definován v úvodní kapitole. Práce se tedy zabývá pouze tímto konceptem.

\section{Typy datových skladů}
Datové sklady se dají rozdělit na několik druhů dle různých kritérií. \citeauthor{Inmon2005}
v knize \citetitle{Inmon2005} z roku \citeyear{Inmon2005} dělí datové sklady podle způsobu a rozsahu,
v jakém data shromažďují a dle prostředí, ve kterém je datový sklad nasazen.

Pro první rozdělovací kritérium, tedy dle rozsahu shromažďovaných dat zavádí Inmon
postupně tři kategorie: hluboký datový sklad \footnote{angl. Deep Data Warehouse}, široký
datový sklad \footnote{angl. Wide Data Warehouse} a hybridní datový sklad \footnote{angl. Hybrid Data
Warehouse}. Tyto jednotlivé kategorie jsou pak definovány takto:
\paragraph{Hluboký datový sklad}
Shromažďovaná data jsou převážně historická a velmi detailní. Slouží primárně k analýze vývoje a trendů.
\paragraph{Široký datový sklad}
Zaměřuje se na velké spektrum informací, přičemž neukládá data v takovém detailu. Využívá se především pro tvorbu přehledů a rychlé vyhledávání dat.
\paragraph{Hybridní datový sklad}
Jedná se o kombinaci obou předchozích kategorií.

 \vspace{5mm}
Pokud datové sklady dělíme dle druhého kritéria, tedy dle prostředí, ve kterém je
nasazujeme, pak Inmon definuje čtyři kategorie, a to Operational Data Stores, Data marts,
Enterprise data warehouse a Virtual data warehouse. Tyto kategorie jsou pak dle Inmona definovány následovně:

\paragraph{Operational Data Stores}
Datový sklad, který je určen k provozním potřebám organizace, pro kterou je zřízen, integruje a centralizuje data z firemních procesů.
\paragraph{Enterprise data warehouse}
Datový sklad, který centralizuje všechna data z celé organizace a slouží jako hlavní znalostní báze pro její řízení.
\paragraph{Data marts}
Malé datové sklady, typicky se budují pro jednotlivá oddělení organizací a na základě jejich specifických potřeb. Tento typ také někdy bývá realizován pouze jako podmnožina velkých datových skladů, jako je Operational Data Store nebo Enterprise data warehouse.
\paragraph{Virtual data warehouse}
Specifický typ datového skladu, který funguje pouze jako transportní a translační vrstva nad různými datovými 
zdroji a pro uživatele tak vytváří jeden přístupový bod. Data tedy nejsou uložena na jednom místě a ve strukturalizované
podobě, ale jsou nahrávána z více externích zdrojů, následně dle potřeby přetransformovaná a doručena uživateli. 

\vspace{5mm}
Každý z těchto typů datových skladů vyžaduje specifický přístup, a to jak
k architektuře samotného skladu, zpracování dat, tak i k technickému vybavení. Z tohoto
důvodu je důležité zvolit správný typ datového skladu již ve fázi návrhu, protože
možnost konverze mezi jednotlivými druhy datových skladů nemusí být vždy možná.

\section{Zpracování dat v datových skladech}
Vzhledem k tomu, že datový sklad většinou slouží jako koncentrátor dat z mnoha různých zdrojů, prochází data při nahrávání transformačním procesem, jehož výstupem jsou strukturovaná data.
Celý tento proces bývá zpravidla automatický, byť to není nikterak podmíněno.

Proces nahrávání dat do datového skladu se nazývá ETL \footnote{z angl. Exract, transform and Load} 
a jedná se o esenciální součást datového skladu. Celý proces se skládá, jak již název
napovídá, ze tří fází, a to fáze extrakce, transformace a finálního nahrání dat do datového skladu.

Kimball a Caserta v knize \citetitle{Kimballc2004} z roku \citeyear{Kimballc2004} celý proces rozšiřují do fází čtyř, a to extrakce, čištění, přizpůsobení \footnote{angl. Conforming}
a doručení. Vzhledem k tomu, že zmíněná publikace patří mezi základní v oboru datových skladů, budu se v další části zabývat jen tímto rozdělením ETL
procesu, a ne pouze základními třemi fázemi. %%tady bych se klidně nebála přidat větu, proč ti to přijde lepší použít čtyři 

\subsection{ETL proces dle Kimballa a Casertyho}
\paragraph{Extract}
První operací, kterou je potřeba pro nahrání dat udělat, je jejich extrahování
z původních úložišť. To probíhá většinou v přesně definovaných cyklech. Během této operace
dojde k nahrání všech požadovaných dat do přechodného úložiště, kde jsou dále
zpracovávány v dalších krocích ETL procesu. Data jsou nahrávána v surové formě a
v původních formátech a souborech, které mohou představovat například data z relačních
databází, XML, csv, JSON nebo XLS soubory.\parencite[s.~18]{Kimballc2004}

Surová data bývají po zpracování většinou smazána, výjimku tvoří například případy,
kdy potřebujeme zachovat dlouhodobou zálohu dat, anebo pokud je potřeba porovnat změny dat mezi
jednotlivými průběh extrakčních cyklů.\parencite[s.~18]{Kimballc2004}

V některých případech je možné využít jako zdroj dat pro datový sklad úložiště jiného
typu, typicky například Data Lake.

\paragraph{Cleaning}
Poté, co jsou požadovaná data vyextrahována je potřeba je vyčistit. Během této fáze
se provádí kontrola integrity dat, odstraňují se duplicity a provádějí další operace s cílem
dosáhnutí požadované datové kvality.\parencite[s.~18-19]{Kimballc2004}

Během této fáze také dobré zvolit požadovanou granularitu, neboli míru detailu
uchovávaných informací \parencite[s.~41]{Inmon2005}, a případně data na požadovanou úroveň
zdecimovat. Granularita patří mezi základní parametry využívané při návrhu datového
skladu, a tak by měla i její hodnota z tohoto návrhu vycházet.\parencite[s.~41]{Inmon2005}

\paragraph{Přizpůsobení (Conforming)}
Fáze potvrzování má význam zejména v případech, kdy extrahujeme data z více
datových zdrojů. Jejím primárním úkolem je data z různých zdrojů propojit dohromady tak,
aby bylo možné se dotazovat napříč všemi těmito zdroji. \parencite[s.~19]{Kimballc2004}

Dalším důležitým úkolem pak je kontrola konfliktů mezi názvy jednotlivých dimenzí, neboť
napříč datovými zdroji nemusí mít vždy stejný význam. \parencite{Kimballc2004}

\paragraph{Doručení (Delivering)}
Během doručovací fáze pak dochází k samotné transformaci dat do stejného formátu a
následně se z dat vytvoří dimenzionální model, nebo požadované schéma. Mezi ty patří například
schéma hvězdice či datová krychle.\parencite[s.~19]{Kimballc2004}

Výstupem této fáze je tedy patřičně setříděný a provázaný soubor dat, nad kterým
je již možné provádět dotazy, jako nad celkem, tedy bez ohledu na datové zdroje a jejich
formát. Takto zpracovaná data se pak následně nahrají do databáze samotného datového
skladu.

Díky ETL procesu jsou ve výsledku v datovém skladu data uložena v jednoduché,
snadno dotazovatelné formě, což značně usnadňuje a urychluje přístup a práci s těmito daty.

\section{Využití datových skladů}
Datové sklady nacházejí velké využití zejména u velkých organizací a společností, ale i
u tak velkých celků, jako jsou například samostatné státy. Velkou výhodou datových skladů je
jejich perzistentní struktura, kdy data můžeme shromažďovat po dlouhou dobu a jejich
analýzu provést až ve chvíli, kdy nastane její potřeba. Díky tomu se hodí na analýzu dat dlouhodobého charakteru.

Jedním z typických zástupců tohoto typu nasazení je pak sdílení dat mezi
vědeckými pracovišti. Například v případové studii \citetitle{Seneviratne20180124} \parencite{Seneviratne20180124} 
se autoři zabývali propojením vědeckých pracovišť Stanford University, Stanford Cancer Institute a California Cancer Registry za účelem vytvoření
společného datového skladu obsahujícího elektronické zdravotní záznamy pacientů
postižených rakovinou prostaty. Díky tomuto datovému skladu vědci mají přístup
k tisícům reálných případů, napříč těmito třemi institucemi.

Další zajímavou aplikací, je například optimalizace a analýza mléčné farmy. Tato
aplikace popsaná v článku \citetitle{Schuetz20180326} \parencite{Schuetz20180326} se zabývá návrhem a implementací
datového skladu, který má za cíl pomocí různých senzorů sledujících například pohyb dojnic
či samotné dojení, optimalizovat a vylepšit výkon této mléčné farmy. Celý projekt je velmi
zajímavý a to zejména z důvodu, že se jedná o specifickou oblast, kde ještě nebyl
koncept datového skladu nasazen.

Zvláštním příkladem jednoúčelově zaměřeného datového skladu je projekt, týkající
se pandemie viru SARS-COV-2. V článku \citetitle{Agapito2020} \parencite{Agapito2020},
autoři popisují implementaci datového skladu vytvořeného za účelem monitorování
šíření viru v Itálii. Zajímavé je, že datový sklad zahrnuje kromě informací týkajících se přímo
samotného viru i další, zdánlivě nesouvisející, informace, které zahrnují například také klimatické informace
jako je znečistěni ovzduší a nebo síla a směr větru. Výzkumné výstupy z tohoto projektu se tak mohou zabývat i vlivem počasí na
šíření nákazy. 

\chapter{ITSM}
Jak již bylo zmíněno v úvodní kapitole, velký rozmach informačních technologií a služeb v 60. letech 20. století si vyžádal vznik patřičných standardizovaných procesů, postupů a pravidel pro řízení služeb v prostředí podniků a organizací. Souhrnně je tento obor označován jako ITSM, neboli Information Technology Service Management, což je pojem, který se původně objevil pouze v rámci metodiky ITIL, ale postupně se rozšířil a zobecnil tak, že v současné době obecně pokrývá problematiku řízení IT služeb.\parencite[s.~20]{Matula2017} 

Matula definuje ITSM jako \textit{souhrn nejlepších praxí a referenčních modelů procesů řízení služeb IT organizace}, přičemž právě přístup, kdy se k IT projektu přistupuje jako k službě znamená, že se daný IT projekt dodává s potřebným personálním a technologickým zajištěním a samotný klient se na chodu dané služby nepodílí a využívá pouze její výstupy.\parencite[s.~20-22]{Matula2017}

\section{Základní terminologie ITSM}
Před samotným vysvětlením základních principů fungování ITSM je nutné nejprve zavést alespoň základní pojmy, které jsou v jeho rámci využívány. Matula ve své publikaci \citetitle{Matula2017} uvádí jako základní tyto následující pojmy, jejichž definice jsou převzaty z \citetitle{SyFvQA11lk1OaIec} z roku \citeyear{SyFvQA11lk1OaIec}:
\paragraph{Nejlepší praktiky (Best practices)}
Osvědčené postupy, které byly využívány ve více organizacích a které se prokázaly jako účinné, efektivní a udržitelné a které vedou k zlepšení výsledků v oblasti kvality, flexibility, nákladů a konkurenceschopnosti.
\paragraph{Služba (Service)}
Prostředek, který zajišťuje doručování hodnoty zákazníkovi, bez nutnosti zapojení zákazníka do řízení nákladů a rizik. Pojem služba může být chápán několika způsoby a to buď jako obecná služba, IT služba, anebo balíček několika služeb. 
\paragraph{Správa (Governance)}
Zajišťuje, aby byly procesy vykonávány správně dle nastavených politik a strategií. Správa zároveň vymezuje role a odpovědnosti, reaguje na zjištěné problémy.
\paragraph{Proces (Process)}
Soubor aktivit, které je třeba vykonat pro dosažení cíle. Aktivity přijímají definované vstupy, které následně za pomoci nástrojů transformují na požadované výstupy.
\paragraph{Funkce (Function)}
Pojem funkce má několik významů. Primárně se jedná o skupinu lidí, nástrojů a jiných zdrojů, které zainteresované osoby využívají k provádění procesů a činností. Dále se pak může jednat o zamýšlený účel konfigurační položky, osoby, týmu nebo procesu. Posledním významem pojmu je pak vyjádření toho, zda je zamýšlený úkol vykonáván správně. 
\paragraph{Role (Role)}
Soubor činností, povinností a pravomocí přidělených konkrétní osobě, anebo skupině osob. Jednotlivec, nebo skupina může mít více rolí.
\paragraph{Kvalita (Quality)}
Schopnost služby, výrobku nebo procesu poskytnout požadovanou hodnotu. Kvalita je ukazatel účinnosti a efektivity procesů.
\section{Základní principy ITSM}
V úvodu této kapitoly již byly využity termíny, jako je proces a nebo kvalita výstupu, které patří do základní terminologie ITSM, a které je potřeba nejdříve definovat. Zde uvedené definice vychází zejména z frameworků, které jsou na ITSM postaveny

Management IT služeb se dle ITSM věnuje několika oblastem, které ovlivňují výslednou kvalitu výstupu a které jsou:
\begin{compactitem}
  \item Lidé, což je oblast zahrnující veškeré lidské zdroje, které jsou potřeba k implementaci a provozu samotné služby, přičemž zahrnuje i uživatele, kteří službu využívají.
  \item Procesy, což je oblast zahrnující veškerou procesní metodiku a postupy, které jsou využívány k řízení a implementaci služeb a to včetně jejich vstupů, výstupů a zodpovědných osob.
  \item Nástroje, které zahrnují veškeré softwarové i hardwarové vybavení, které umožňuje dané procesy zefektivnit či zautomatizovat.
\end{compactitem}

V rámci poskytování služeb je také důležité hlídat kvalitu výstupů. Za tímto účelem definuje ITSM základní indikátory, které umožňují kvalitu služby měřit a následně vyhodnotit. Konkrétně se pak jedná o tyto složky IT služeb \parencite[s.~20]{Matula2017}:
\begin{compactitem}
\item Růst a hodnota jsou ukazatele, které se zabývají změnou výnosů před a po zavedení dané služby.
\item Rozpočet a jeho dodržování, jakožto ukazatelů zabývajících se optimalizací projektového rozpočtu s cílem eliminovat zbytečné výdaje.
\item Riziko dopadu, které indikuje a vyhodnocuje následky případných rizik.
\item Efektivita a komunikace, vyhodnocující zpětnou vazbu od zákazníků a jejich spokojenost s poskytovanými službami.
\end{compactitem}

ITSM je v současnosti popsán normou ISO 20000 Informační technologie - Management služeb IT a na jejímž základě existuje mnoho různých frameworků a knihoven nejlepších praktik. Patří mezi ně například COBIT, FitSM a zejména pak ITIL.\parencite[s.~25]{Matula2017}

\begin{figure}[h]
  \begin{center}
          \includegraphics[width=10cm]{img/itsm-diag.drawio.png}
  \end{center}
  \caption{Řízení systémů IT \parencite[s.~21]{Matula2017}}
  \label{fig:itsmDiag}
\end{figure} 

Souhrnné schéma ITSM je vyobrazené na obrázku \ref{fig:itsmDiag}, ze kterého je patrné, že samotné ITSM je postaveno na čtyřech pilířích. Základní dvojici tvoří vstupy a výstupy. Díky vstupům může poskytovatel dodat zákazníkovi potřebné výstupy, neboli hodnotu. Tento proces je pak ovlivňován prostředím, ve kterém je služba realizována, a které do značné míry ovlivňuje způsob realizace. Poslední částí je pak řízení určující metodiku, kterou je služba realizována. 


\section{ITIL}
Jednou z nejrozšířenějších knihoven v oblasti ITSM je knihovna ITIL \footnote{z angl. Information Technology Infrastructure Library}, která je v současnosti spravována britskou společností Axelos Ltd. \parencite[s.~31]{Matula2017}. Tato knihovna byla dle průzkumu Forbes Insight z roku 2017 částečně implementována alespoň ve 47\% zkoumaných organizací\parencite{Watts3082017} a byla využita i při  zavádění datového skladu na Masarykově univerzitě. 

ITIL vznikl v 80. letech 20. století na objednávku britské vlády a původně byl nazýván GITIM \footnote{z angl. Government Information Technology Infrastructure}. Během 90. let byl dobře přijímán velkými organizacemi a vládami, což z něj učinilo standard na poli ITSM knihoven a frameworků, a i nadále slouží jako základ pro nové frameworky, které jsou z něj odvozeny, jako je například MOF\footnote{z angl. Microsoft Operations Framework} od společnosti Microsoft. \parencite[s. ~31]{Matula2017}

Knihovna ITIL je průběžně aktualizována a momentálně nejrozšířenější jsou verze 3 z roku 2007 a verze V4 z roku 2017. Mezi těmito verzemi došlo k velké změně pojetí. 

Třetí verze frameworku se ITIL se věnuje převážně samotnému procesu zavádění služby, od jejího návrhu, přes realizaci, až po každodenní využití služby. Jejím základním cílem je optimalizace procesů IT oddělení společností a organizací tak, aby byly v souladu s jejich obchodními cíli. Primárně má za úkol zajistit, aby byly realizované služby řízeny právě obchodními požadavky a samotné IT oddělení má za cíl pouze jejich implementaci.\parencite[s.~8]{Carlidge2007} Oproti tomu ITIL ve verzi V4 se primárně věnuje doručované kvalitě, nákladům a rizikům. 

Zjednodušeně řečeno ITIL ve verzi 3 cílí na fungování dodavatele, coby pouhého realizátora služby a na proces vedoucí k jejímu poskytování. Oproti tomu ITIL verzi 4 se zaobírá především důvody a cíli, které ke vzniku služby vedly.\footnote{https://www.alvao.com/cs/blog/dlouha-cesta-k-itil4-jak-nam-historie-itil-pomuze-lepe-ridit-it se pak rozhodni jestli to chceš ocitovat a jak}
Z tohoto důvodu jsou obě verze detailně popsány v následujících kapitolách.  
\subsection{ITIL V3/ITIL 2011}
Jak již bylo uvedeno v předchozí kapitole, knihovna ITIL ve verzi 3 byla představena v roce 2007. V roce 2011 však prošla aktualizací a bývá tak označována jako ITIL 2011. Aktualizace se týkala především zavedení nových procesů a prohloubení definic stávajících procesů a pojmů. Nicméně i přes tuto aktualizaci zůstávají verze V3 a 2011 kompatibilní.\parencite{Kempter2722013}

Jak je znázorněno na obrázku \ref{fig:itil3_lifecycle}, knihovna ITIL 2011 se věnuje celistvému procesu managementu služeb, od stanovení obchodní strategie, přes její vývoj a nasazení do provozu až po kontinuální vylepšování. 
  
\begin{figure}[h]
  \begin{center}
          \includegraphics[width=10cm]{img/itil_V3_proces.png}
  \end{center}
  \caption{Životní cyklus ITIL 2011 \parencite[s.~7]{Carlidge2007}}
  \label{fig:itil3_lifecycle}
\end{figure}  

Na obrázku \ref{fig:itil3_schema} je znázorněno schéma celé knihovny ITIL 2011, která je členěna na pět fází, které se postupně věnují celému procesu managementu služeb - od obchodní analýzy a strategie, přes její design, nasazení a správu až po její následný rozvoj. Ze schématu je také patrné, že fáze se skládají z jednotlivých procesů a funkcí, které slouží k samotné realizaci.

\begin{figure}[h]
  \begin{center}
          \includegraphics[width=11cm]{img/itil_V3_schema.png}
  \end{center}
  \caption{Schéma ITIL 2011 \parencite[s.~32]{Matula2017}}
  \label{fig:itil3_schema}
\end{figure} 


Každá fáze, která je vyobrazena na obrázku \ref{fig:itil3_schema} je v knihovně ITIL 2011 zastoupena samostatnou publikací. Ta problematiku dané fáze do detailu rozebírá a popisuje jednotlivé procesy a funkce. V rámci knihovny ITIL 2011 je zastoupena samostatnou publikací. 
\paragraph{Strategie služeb}
První publikace se zabývá návrhem strategie nové služby tak, aby primárně řešila daný obchodní problém s důrazem na kvalitu výstupu. Tato kvalita musí vycházet z požadavků klienta, pro kterého je služba navrhována a poskytována, ale zároveň musí nové služby být plně kompatibilní s interními procesy jak poskytovatele služby, tak i s procesy samotného klienta.\parencite[s.~12-13]{Carlidge2007}

Pro nastavení kvalitní strategie služby je nutné správně určit odpovědi na několik základních otázek, které následně slouží jako základní parametry pro vznikající strategii.\parencite[s.~32]{Matula2017} Mezi tyto základní otázky patří například následující:

\begin{compactitem}
  \item Jaké služby nabízet
  \item Kdo je jejich možný odběratel
  \item Co je hlavní doručovanou kvalitou realizované služby
  \item Jakým způsobem budou služby finančně řízeny a kontrolovány
  \item Metodika určování možných vylepšení služby a jejich prioritizace
  \item Jak robustní má služba být, aby byly dostatečně ochráněny investované zdroje a manažerské kapacity
  \item Jakým způsobem se bude měřit efektivita a výkon dané služby
\end{compactitem}

Po zodpovězení těchto základních otázek, je následně při návrhu služeb nutné dodržet princip tz. čtyř P, který je realizován pomocí následující bodů:\parencite[s.~13]{Carlidge2007}
\begin{compactitem}
    \item Perspektiva - určuje charakteristické vize a směr dané strategie
    \item Pozice - udává základ, ze kterého má poskytovatel služby vycházet
    \item Plán - říká jakým způsobem má být daných cílů dosaženo
    \item Předloha - určuje základní postup rozhodování a realizace jednotlivých kroků a řešení problémů
\end{compactitem}

Kvalitní zpracování konceptu čtyř P zaručuje, dostatečně kvalitní základ pro další fáze managementu služeb.


\paragraph{Design služeb}
Druhá publikace se zabývá metodikami návrhu a změn samotných služeb, a to tak aby byla zajištěna správná funkcionalita služby s ohledem na požadované obchodní cíle a jejich případnou změnu. Za tímto účelem poskytuje publikace dva základní postupy. Prvním je správné vedení designu, vývoje služeb a praktik určených k jejich managementu. Druhým jsou pak základní metody a principy konverze strategických cílů v samotné službě.\parencite[s.~21]{Carlidge2007}

Za tímto účelem publikace definuje několik aspektů a principů. Prvním z nich je petice základních aspektů designu služeb, které jsou: \parencite[s.~22]{Carlidge2007}
\begin{compactitem}
    \item Řešení nově vznikajících a měnících se služeb
    \item Management informačních systémů a nástrojů
    \item Technologie a architektura managementu služby
    \item Procesy stanovené v rámci poskytování služby
    \item Měřící metriky a metody
\end{compactitem}
Pomocí těchto klíčových aspektů je zaručen holistický přístup a konzistence navrhovaných služeb v rámci organizace a správná integrace procesu v rámci IT oddělení a celé organizace, která dané služby poskytuje. \parencite[s.~22]{Carlidge2007}

Druhým klíčovým principem jsou čtyři P designu služeb, jejcihž cílem je zajistit dobrou efektivitu služeb. Mezi tyto klíčové body patří:\parencite[s.~22]{Carlidge2007}
\begin{compactitem}
    \item Lidé (People), kteří jsou zainteresování v poskytování dané služby
    \item Produkty které zastupují technologie a spravované systémy, které slouží k doručení služby
    \item Procesy, role a aktivity, které slouží k poskytování služby
    \item Partneři, kteří zastupují dodavatele, výrobce a prodejce, kteří zajišťují dané službě podporu
\end{compactitem}

Posledním základním principem, které jsou v publikaci definovány je  \emph{Balíček designu služeb} (z angl. Service Design Package, zkráceně SDP), který na základě dříve zmíněných hledisek a principů, definuje všechny aspekty služby a požadavků na ni v každé fázi jejího životního cyklu. SDP je vytvářeno s každou novou IT službou, aktualizací a nebo ukončením jejího poskytování.\parencite[s.~23]{Carlidge2007}
\paragraph{Přechod služeb}
Obsahem třetího svazku je popis správné metodiky, která má za úkol zajistit doručení nových a upravených služeb a služeb, u kterých bylo poskytování již ukončeno, a to s ohledem na naplněni obchodních cílů vycházeních ze strategie a designu služeb ustanovených dle první a druhé publikace.\parencite[s.~30]{Carlidge2007}

Klíčovým prvkem této fáze životního cyklu služeb je tedy nastavení správných manažerských procesů, které se zabývají nasazením služby do provozu, rizikového managementu, nastavení očekávaných výstupů, testování, přenosu znalostí týkajících se služeb atp.\parencite[s.~30]{Carlidge2007}

V rámci přechodu služby je nutné zadefinovat základní parametry, které umožňují službu správně zrealizovat. Tyto jsou zastoupeny následujícími body:\parencite[s.~30]{Carlidge2007}
\begin{compactitem}
    \item Potenciální obchodní hodnota, kdo jí doručuje a vyhodnocuje
    \item Identifikace všech zainteresovaných osob, které mají na úspěšný přechod vliv
    \item Implementace a případná adaptace designu služby, pokud se během přechodu ukáže nutnost změn
\end{compactitem}

Na základě těchto tří parametrů je následně možné realizovat přechod, který stojí na následujících základních principech, které zaručují efektivní efektivní přenos nové nebo modifikované služby. \parencite[s.~30]{Carlidge2007}
\begin{compactitem}
    \item Správné pochopení všech aspektů, principů, záruk a výstupů podpory služby
    \item Dobře zmapovaná komplexita technologických změna  procesů služby
    \item Ustavení manažerského rámce zavádění změn za účelem zajištění správného doručení výstupů a zvážení všech rizik
    \item Podpora přenosu znalostí a rozhodovacích kritérií týkající se všech procesů a funkcí služby a jejich správný přenos mezi všemi zainteresovanými stranami
    \item Ustanovení procesů které mají za úkol předvídat potřebné změny a jejich správné načasování
    \item zajištění zapojení veškerých stran zapojených do přechodu služby a jejich zaškolení do všech aspektů týkajících se přechodu služby
\end{compactitem}

Přenos služby spravovaný dle těchto pravidel zajišťuje bezproblémové uvádění nových a modifikovaný služeb do provozu a stejně tak i jejich ukončení. 
\paragraph{Správa služeb}
Čtvrtou fází životního cyklu služby dle ITIL 2011 je správa provozu samotné služby a jejím primárním cílem je zajištění, že služba bude doručovat nasmlouvanou hodnotu v požadované kvalitě.\parencite[s.~36]{Matula2017}

Tato fáze je oproti předchozím mírně specifická, a to jednak tím, že se prakticky výhradně zabývá uživateli služby a dále pak tím, že kromě procesů jsou v rámci této publikace definovány čtyři základní funkce, které je nutné v rámci správné implementace správy služeb dle ITIL 2011, zavést. Tyto základní funkce jsou: \parencite[s.~46]{Carlidge2007}
\begin{compactitem}
    \item Service desk - poskytuje uživatelům jednotný kontaktní bod, přes který je možné zaznamenávat změnové požadavky, incidenty, přístupové požadavky atp. 
    \item Technický management - umožňuje management správců technické infrastruktury, který pomáhá plánovat, implementovat a u udržovat stabilní technické prostředí pro správné fungování služby.
    \item Aplikační management - zajišťuje správu všech aplikací, které služba využívá, často je tato funkce využívána napříč ruznými službami, neboť samotné aplikace mohou být využívaný více službami.
    \item Řízení provozu IT - spravuje IT infrastrukturu služeb. Výše zmíněný technický management se nevěnuje primárně IT infrastruktuře, nýbrž obecně technickému zajištění služby. Tato funkce má zadefinovány také dvě podružné funkce, které se věnují personálnímu zajištění a správě fyzických zařízení, jako jsou typicky například data centra.
\end{compactitem}
\paragraph{Průbežné zlepšování služby}
Pátý a poslední svazek se zabývá průběžným vylepšováním služeb tak, aby byla vždy doručená požadovaná hodnota, která se může v průběhu času měnit. Vzhledem k tomu, že tyto změny probíhají na již fungující službě, je v rámci jejich návrhu, vývoje a nasazení využíván mechanismus PDCA\footnote{Zkratka prvních písmen anglických slov Plan, Do, Check a Act} neboli Demingův cyklus. Ten proces zavádění průběžných změn dělí do čtyřech následujících fází: \parencite[s.~38]{Matula2017}
\begin{compactitem}
    \item Plánuj - Určení strategie, která má vést k zlepšení a definování potřebných metrik
    \item Dělej - Sběr a zpracovaní potřebných dat
    \item Kontroluj - Analýza nasbíraných informací a údajů, jejich prezentace a využití
    \item Jednej - Aplikace vylepšení dané služby
\end{compactitem}
Mechanismus PDCA tímto zajišťuje, že je služba neustále hodnocena z hlediska doručované hodnoty a její kvality, čímž je zajištěn její postupný vývoj po malých krocích, jejichž realizace typicky bývá méně náročná i riziková. 

Tyto publikace dohromady tvoří základní korpus samostatné knihovny a jsou dále doplněny dalšími publikacemi a informačními zdroji, které se dané problematice věnují více do hloubky. \parencite[s.~8]{Carlidge2007}

\subsection{ITIL V4}
Nejnovější verze frameworku ITIL, tedy verze 4, byla uvedena v roce 2019. Jak již bylo uvedeno úvodu kapitoly, mezi verzemi V3 respektive 2011 a verzí 4 došlo k velmi výrazným změnám. Čtvrtá verze knihovny ITIL, je oproti svým předchůdcům založená velmi holisticky a její primárním zaměřením není samotný proces, kterým je služba realizována a poskytována, ale zaměřuje se primárně doručenou hodnotu a její kvalitu a to zejména z hlediska uživatele uživatele služeb. Tato verze knihovny již také implementuje moderní metody vývoje, jako je Lean, Agile a nebo DevOps.\footnote{Tohle možná nějak rozveď, nebo alespoň prolinkuj na nějakej slovník} \parencite[s.~7]{Cartlidge2020}

ITIL V4 na rozdíl od předchozí verze nepracuje s jednotlivými fázemi realizace služby a místo toho zavádí nový koncept zvaný \emph{Service Value System}\footnote{zkráceně SVS}, který se skládá ze čtyř vrstev, které jsou definovány v další části této kapitoly. SVS vychází ze čtyři základních dimenzí managementu služeb, jejichž schéma je zobrazeno na obrázku \ref{fig:itil4_dimension}. Tyto základní dimenze poskytují holistický přístup k celému managementu služeb a ovlivňují všechny aspekty SVS.\parencite[s.~10]{Cartlidge2020} Definice jednotlivých dimenzí je pak následovná:

\begin{figure}[h]
  \begin{center}
          \includegraphics[width=11cm]{img/ITIL_V4_dimesions.png}
  \end{center}
  \caption{Čtyři základní dimenze managementu služeb \parencite[s.~32]{Cartlidge2020}}
  \label{fig:itil4_dimension}
\end{figure} 

\begin{compactitem}
    \item Organizace a lidé (Organizations and people) - První dimenze se věnuje primárně samotné správě organizace a vnitřní struktuře jejího personálního zajištění, dále může zahrnovat i samotné zákazníky a odběratele služeb a další zainteresované osoby nebo dodavatele. Všechny takto zainteresované strany musí být začleněny do jasné hierarchie, která reflektuje jejich schopnosti a cíle. 
    \item Informace a technologie (Information and technologies) - Tato dimenze se věnuje informacím a znalostem, které jsou potřeba k úspěšnému poskytovaní služby. Informace a znalosti jsou shromažďovány s ohledem a s důrazem na technologie, které jsou využívány v rámci správy služeb. Mezi takto spravované zdroje tak patří například databáze, komunikační systémy, cloudová prostředí, ale i například vytvořená, spravovaná a archivovaná data.
    \item Partneři a dodavatelé (Partners and suppliers) - V rámci třetí dimenze jsou řešeny vztahy mezi partnery a dodavateli, na kterých bývá realizace většiny služeb závislá, jelikož málo kdy jsou služby absolutně nezávislé na externích subjektech. V rámci těchto externích subjektů je nutné definovat hloubku propojení a integrace mezi nimi. Klíčovými faktory je pak správná kooperace těchto subjektů a to z hlediska nákladů, sdílení vědomostí a požadavků.
    
    \item Hodnotové proudy a procesy (Value streams and processes) - Čtvrtá dimenze definuje činnosti, pracovní postupy, procesy a jejich kontrolu tak, aby došlo k dosažení požadovaných cílů. Primárním oborem zájmu je tak v tomto případě hlavně zajištění komunikace a kolaborace mezi jednotlivými částmi organizace, která má za úkol poskytování dané služby, dále se pak dimenze zabývá tzv. hodnotovým tokem, který je definován jako \emph{řada kroků, které je třeba vykonat aby došlo k vytvoření a dodání požadované hodnoty klientům.}
\end{compactitem}

Mimo tyto hlavní dimenze je, vzhledem k tomu, že služby nejsou provozovány v uzavřeném prostředí, nutné zohlednit i externí faktory, které samotný design, vývoj i provoz ovlivňují a které se v určitém ohledu dají brát jako jedna ze základních dimenzí, jelikož je nikdy není možné úplně vyloučit. Na schématu \ref{fig:itil4_dimension} jsou vyznačeny jako vnější obálka zahrnující politické, ekonomické, sociální, technologické, právní a enviromentální faktory, zkráceně se pro tyto faktory používá zratka PESTLE.\footnote{z angl. Political, Economic, Social, Technological, Legal and Enviroment}

\vspace{5}
Jak již bylo zmíněno, tak na výše definovaných základních dimenzích je postaven nový systém správy služeb, který se nazývá Service Value System, neboli SVS. Tento systém se skládá z několika vrstev, jejichž schéma je znázorněno na obrázku \ref{fig:SVS_schema}. Ze schématu vyplývá, že prvotními vstupy jsou požadavky nebo příležitosti, které postupně procházejí jednotlivými vrstvami přes hlavní principy managementu služby, její správu, řídící postupy až po průběžné vylepšení, ústřední části je pak \emph{Service value chain}. Na konci toho procesu je výstupem hodnota v požadované kvalitě. \parencite[s.~14]{Cartlidge2020} Významy a obsah jednotlivých vrstev, je detailně rozepsán níže. 

    \begin{figure}[b]
        \begin{center}
            \includegraphics[width=11cm]{img/SVS_schmea.png}
        \end{center}
        \caption{Schéma SVS \parencite[s.~21]{Cartlidge2020}}
        \label{fig:SVS_schema}
    \end{figure} 
\begin{compactitem}
    \item Hlavní principy (Guiding principles) - První vrstva se zabývá komplexním řízením celé organizace a za všech situací a vytváří základy pro interní kulturu poskytovatele, rozhodovací mechanismy a celkově proces pro řízení poskytování služeb. Tato vrstva je rozdělena na sedm klíčových principů, které pokrývají celý proces poskytování služeb. 
    \begin{compactitem}
      \item Zaměř se na hodnotu - Nejdůležitější je doručená hodnota a její kvalita, ať pro uživatele služby, tak pro samotného dodavatele. Dáležitým ukazatelem kvality služby je v tomto případě tzv. Customer Experience\footnote{zkráceně CX}, který udává jak je uřivatel s danou službou spokojený.
      \item Začínej v aktuální situaci - není nutné vždy začít vyvýjet slubu od začátku, je vhodné stavět na již existujích prostředcích, prostředcích, službách atp., neboť vývoj od samého začátku je velmi náročný na zdroje a z pravidla zahrnuje i již jednou řešené situace. 
      \item Postupuj iterativně a s ohledem na odezvu - Velké úkoly je nutné rozdělit do menších celků, které se dají lépe managovat a to jednak v ohledech kontroly jejich samotné realizace, tak s ohledem na snazší řešení případných problémů. Jednotlivé úkoly by měly být zaváděny postupně a s ohledem na odezvu, neboť se v průbehu vývoje či poskytování mouhou požadavky na službu lišit. 
      \item Collaborate and promote visibility - Personál musí spolupracovat a zajišťovat, aby kažný, kdo je zainteresovaný do procesu poskytování služby, měl přiřazenou patřičnou roli, která odpovídá jeho schopnostem. Je důležité, aby byla pracovní náplň jednotlivců či týmů známá, neboť je tak možné snáze hledat chyby a případně změny v projektu a zjednat jejich nápravu.
      \item Mysli a pracuj holisticky - Žádný ze zdrojů potřebných pro realizace a poskytování služby není od ostatních izolován a nejedná samostatně. Cílem je, aby v celém procesu nevznikala slepá a neznámá místa, celý procest musí přesně zmapován od požadavku do doručení kvality.
      \item Udrž to jednoduché a praktické - Cílem je využívat co nejjednodušší postupy a zdroje. Je důležité uchovávat v provozu pouze ty zdroje, procesy, metriky atp., které přinášejí požadovaný výstup. 
      \item Optimalizuj a automatizuj - V rámci organizace je důležité automatizovat rurinní úkoly v co největší míře, což vede k úposře nákladů a zpravidla i k menší chybovosti. Zautomatizované úkoly je nutnbé udržovat v optimalizované formě, neboť neoptimalizované procesy mohou naopak oraganizaci zatěžovat.
    \end{compactitem} 
    \item Správa (Governance) - Druhá vrstva se věnuje nastavení potřebných procesů pro evaluaci, monitoring a directs \footnote{jak to přeložit?} uvnitř organizace poskytující služby. Může se jednat o celou organizaci a nebo jen její dílčí části, cílem je zajištění správného směřování vůči cílům a prioritám organizace. 
    \item Hodnotový řetězec služby (?) - Jak již bylo zmíněno výše, jedná se o soubor kroků, které je třeba provést, aby z dodaných vstupů vznikla požadovaná hodnota. Tento řetězec se skládá z šesti samostatných aktivit a celé schéma hodnotového řetězce je vyobrazeno na obrázku \ref{fig:value_chain}, jeho jednotlivé části jsou pak definovány následovně:
    \begin{compactitem}
        \item Plan - Zajišťuje sdílení jednotné vize, stavu a směru dalšího vývoje produktu či služby.
        \item Zlepšení - Věnuje se postupnému zlepšování služby, nebo produktu.
        \item Zapojení - Zajišťuje správné pochopení potřeb všech zainteresovaných stran, jejich transparentnost a kontinuální zapojení. 
        \item Design a přechod - Zprostředkovává kontrolu kvality výsledného produktu/služby, nákladů a jejich uvedení na trh.
        \item Obtain/Build(?) - Zabezpečuje, že jsou jednotlivé komponenty služby/produktu dostupné, když jsou potřeba a dle potřebných specifikací.
        \item Doručení a podpora - Zaručuje doručení produktu/služby dle domluvených specifikací a očekávání zainteresovaných stran.
    \end{compactitem}
    Hodnotový řetězec služby je značně flexibilní a umožňuje nasazení v rámci nejrůznějších vývojových přístupů. Díky této flexibilitě umožňuje velmi snadno, rychle a efektivně reagovat na případné změnové požadavky zainteresovaných stran. Spolu s praktikami ITIL tak tvoří velmi komplexní sadu nástrojů pro správu IT služeb.\parencite[s.~21]{Cartlidge2020}

    \begin{figure}[b]
        \begin{center}
            \includegraphics[width=11cm]{img/value_chain.png}
        \end{center}
        \caption{Service Value Chain \parencite[s.~21]{Cartlidge2020}}
        \label{fig:value_chain}
    \end{figure} 
    
    \item Postupy (Practices) - V rámci třetí vrstvy je definována řada postupů, které jsou designovány dle čtyř základních dimenzí managementu služeb a každá podporuje více hodnotových řetězců. Tyto postupy se dělí do tří kategorií a to \emph{Obecné manažerské postupy}, \emph{Postupy managementu služeb} a \emph{Postupy technického managementu}, jejich kompletní seznam je pak uveden v tabulce \ref{tab:management_practices_ITIL}.
    Tyto postupy jsou navrženy tak, aby zajistili správné pochopení a ukotvení celkového přehledu tykajícího se struktury, obsahu a klíčových konceptů uživateli služby nebo produktu. Z tohoto jsou všechny vystavěny dle následujícího schématu: \parencite[s.~23]{Cartlidge2020}
    \begin{compactitem}
        \item Obecné informace
        \begin{compactitem}
            \item Účel a popis
            \item Termíny a pojmy
            \item Rozsah/záběr
            \item Faktory úspěšné praxe
            \item Klíčové metriky
        \end{compactitem}
        \item Hodnotové proudy a procesy
        \begin{compactitem}
            \item Jak praxe přispívá k činnostem hodnotového řetězce služeb
            \item Procesy a činnosti praxe
        \end{compactitem}
        \item Organizace a lidé
            \begin{compactitem}
                \item Role, kompetence a zodpovědnosti
                \item Organizační struktury a týmy
            \end{compactitem}
        \item Informace a technologie
        \begin{compactitem}
            \item Výměna informací: vstupy a výstupy
            \item Automatizace a nástroje
        \end{compactitem}
        \item Partněři a dodavatelé
        \begin{compactitem}
            \item Vztahy s třetími stranami zapojenými do praxe
            \item Zohlednění zdrojů
        \end{compactitem}
    \end{compactitem}
    
    \item Průběžné vylepšení (Continual improvement) - Model kontinuálního vylepšování je v rámci SVS definován jako cyklický řetězec otázek, které jsou úzce propojeny s jednotlivými procesy a metrikami, díky kterým je možné tyto otázky zodpovědět a provést případná vylepšení. Schéma celého řetězce je znázorněno na obrázku \ref{fig:svs_continual_improvement}.

    \begin{figure}[b]
        \begin{center}
            \includegraphics[width=11cm]{img/continual_improvement.png}
        \end{center}
        \caption{Model průběžného vylepšování dle ITIL \parencite[s.~24]{Cartlidge2020}}
        \label{fig:svs_continual_improvement}
    \end{figure} 
\end{compactitem}

    \begin{table}[b]
        \scriptsize
        \begin{tabularx}{\textwidth}{100X}
            \toprule
                Obecné manažerské postupy & Postupy managementu služeb & Postupy technického managementu \\
             \midrule
                Architecture management & Availability management & Deployment management \\
                Continual improvement & Business analysis & Infrastructure and platform management \\
                Information security management & Capacity and performance & Software development and management \\
                Knowledge management & Change enablement  \\
                Measurement and reporting & Incident management  \\
                Organizational change management & IT asset management  \\
                Portfolio management & Monitoring and event management  \\
                Project management & Problem management  \\
                Relationship management & Release management \\
                Risk management & Service catalogue management \\
                Service financial management & Service configuration management \\
                Strategy management & Service continuity management \\
                Supplier management & Service design \\
                Workforce and talent management & Service desk \\
                Service level management \\
                Service request management \\
                Service validation and testing \\
            \bottomrule
        \end{tabularx}
        \caption{Manažerské postupy dle ITIL \parencite[s.~22]{Cartlidge2020}}
        \label{tab:management_practices_ITIL}
    \end{table}

\chapter{DWH v rámci Masarykovy univerzity}
\section{Specifikace, výhody}
dasdasd

\section{Zpracovávaný usecase - provozně ekonomická data}
dasdasd

\section{Evaluace a optimalizace procesu}
sadasdasd

\chapter{Aplikační část}
dasdasd

\chapter{Zpracovávaný usecase}
asdasd

\section{Technické řešení}
\subsection{Technické řešení DWH}
asdasd
\subsection{Usecase}
sdas

\section{Návrh implementačního procesu}
\section{Zavedení}

\chapter{Závěr}
\section{Diskuse výsledků práce}
\subsection{Teoretická část}
\subsection{Aplikační část}
\section{Konec, finito, čus bus}

\printbibliography[heading=bibintoc] %% Print the bibliography.

  \makeatletter\thesis@blocks@clear\makeatother
  \phantomsection %% Print the index and insert it into the
  \addcontentsline{toc}{chapter}{\indexname} %% table of contents.
  \printindex

\appendix %% Start the appendices.
\chapter{An appendix}
Here you can insert the appendices of your thesis.

\end{document}
